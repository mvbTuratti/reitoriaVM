
\documentclass[thesis]{hmcposter}
\usepackage{graphicx}
\usepackage{natbib}
\usepackage{booktabs}
\usepackage{subfig}
\usepackage{amsmath}
\usepackage{textcomp}
\usepackage{longtable}
\usepackage{url}
\posteryear{2020}
\author{ } 
\title{ADMIN}
\class{Calendário Acadêmico 2020}
\pagestyle{fancy}
\usepackage{setspace}
\onehalfspacing
\begin{document}
\begin{poster}
\normalsize\section{\color{hmcorange}Graduação}\subsection{Março \hfill 26 dias letivos}\textbf{07}\qquad \textbf{Início das Atividades Acadêmicas para os Calouros} \newline \null\textbf{10}\qquad Início do período para solicitação de \textbf{exame de suficiência} em disciplinas/unidades curriculares \newline \null\textbf{10}\qquad Início do período para solicitação de \textbf{equivalência ou convalidação} de disciplinas/unidades curriculares \newline \null\textbf{10}\qquad Data limite para solicitação de \textbf{exame de suficiência} em disciplina/unidade curriculares \newline \null\textbf{11}\qquad \textbf{Início do semestre letivo} \newline \null\textbf{19}\qquad Testando Inclusão \newline \null\subsection{Abril \hfill 22 dias letivos}\vfill\null
\columnbreak
\section{\hfill \color{hmcorange}1º Semestre}
\subsection{Maio \hfill 18 dias letivos}\textbf{01}\qquad Dia da mentira \newline \null\textbf{11}\qquad Data limite para \underline{publicação} da \textbf{primeira nota bimestral} e \underline{lançamento} no sistema acadêmico \newline \null\subsection{Junho \hfill 23 dias letivos}\textbf{03}\qquad Data limite para divulgação dos resultados do \textbf{exame de suficiência} em disciplinas/unidades curriculares \newline \null\newpage
\section{\color{hmcorange}Graduação}\subsection{Julho \hfill 12 dias letivos}\textbf{18}\qquad \textbf{Término do semestre letivo} \newline \null\textbf{18}\qquad Data limite para \underline{publicação} da \textbf{segunda nota bimestral} e \underline{lançamento} no sistema acadêmico \newline \null\textbf{19}\qquad Data limite para o \underline{envio} dos \textbf{diários de classe} ao DERAC \newline \null\subsection{Agosto \hfill 19 dias letivos}\textbf{08}\qquad \textbf{Início das Atividades Acadêmicas para os Calouros} \newline \null\textbf{12}\qquad \textbf{Início do semestre letivo} \newline \null\textbf{12}\qquad Início do período para solicitação de \textbf{exame de suficiência} em disciplinas/unidades curriculares \newline \null\textbf{26}\qquad Início do período para solicitação de \textbf{equivalência ou convalidação} de disciplinas/unidades curriculares \newline \null\subsection{Setembro \hfill 24 dias letivos}\textbf{20}\qquad Data limite para solicitação de \textbf{exame de suficiência} em disciplinas/unidades curriculares \newline \null\vfill\null
\columnbreak
\section{\hfill \color{hmcorange}2º Semestre}
\subsection{Outubro \hfill 20 dias letivos}\textbf{14}\qquad Data limite para \underline{publicação} da \textbf{primeira nota bimestral} e \underline{lançamento} no sistema acadêmico \newline \null\subsection{Novembro \hfill 24 dias letivos}\textbf{04}\qquad Data limite para divulgação dos resultados do \textbf{exame de suficiência} em disciplinas/unidades curriculares \newline \null\subsection{Dezembro \hfill 15 dias letivos}\textbf{19}\qquad \textbf{Término do semestre letivo} \newline \null\textbf{19}\qquad Data limite para \underline{publicação} da \textbf{segunda nota bimestral} e \underline{lançamento} no sistema acadêmico \newline \null\textbf{20}\qquad Data limite para o \underline{envio} dos \textbf{diários de classe} ao DERAC \newline \null\textbf{31}\qquad semestreee \newline \null\newpage
~
\vfill
\begin{center}
\large \textbf{DIAS LETIVOS DOS CURSOS DE GRADUAÇÃO}
\newline
\null
\newline
\begin{table}
\centering
\resizebox{0.7\columnwidth}{!}{
\begin{tabular}{|c|c|c|c|c|c|c|}
\hline
\textbf{SEMESTRE} & \textbf{SEG} & \textbf{TER} & \textbf{QUA} & \textbf{QUI} & \textbf{SEX} & \textbf{SÁB} \\ \hline
1º & 18 & 18 & 18 & 17 & 15 & 15 \\ \hline
2º & 16 & 18 & 17 & 18 & 17 & 17 \\ \hline
\multicolumn{6}{|c|}{\small \textbf{Total de dias efetivos de atividades acadêmicas:}}              & 210            \\ \hline
\end{tabular}
}
\end{table}
\null
\end{center}
\vfill
\null
\columnbreak
~
\vfill
\begin{center}
\large \textbf{DATAS IMPORTANTES PARA OS CURSOS DE GRADUAÇÃO}
\newline
\null
\newline
\begin{table}
\centering
\resizebox{0.75\columnwidth}{!}{
\begin{tabular}{|c|c|c|}
\hline
\textbf{SEMESTRE} & \textbf{\begin{tabular}[c]{@{}c@{}}TÉRMINO\\ DO SEMESTRE\end{tabular}} & \textbf{\begin{tabular}[c]{@{}c@{}}RESULTADOS\\ FINAIS\end{tabular}} \\ \hline
1º & 14 de Julho & 14 de Julho \\ \hline
2º & 17 de Dezembro & 17 de Dezembro \\ \hline
\end{tabular}
}
\end{table}
\newline
\null
\newline
Os cronogramas de matrícula serão divulgados em instrução própria e publicados no portal dos alunos
\end{center}
\vfill
\null
\newpage\section{\color{hmcorange}Calem}\subsection{Março \hfill 26 dias letivos}\subsection{Abril \hfill 20 dias letivos}\vfill\null
\columnbreak
\section{\hfill \color{hmcorange}1º Semestre}
								\subsection{Maio \hfill 18 dias letivos}\subsection{Junho \hfill 23 dias letivos}\newpage
\section{\color{hmcorange}Calem}\subsection{Julho \hfill 12 dias letivos}\subsection{Agosto \hfill 19 dias letivos}\subsection{Setembro \hfill 24 dias letivos}\vfill\null
\columnbreak
\section{\hfill \color{hmcorange}2º Semestre}
							\subsection{Outubro \hfill 20 dias letivos}\subsection{Novembro \hfill 24 dias letivos}\subsection{Dezembro \hfill 15 dias letivos}\newpage
~
\vfill
\begin{center}
\large \textbf{DIAS LETIVOS DOS CURSOS DO CALEM}
\newline
\null
\newline
\begin{table}
\centering
\resizebox{0.7\columnwidth}{!}{
\begin{tabular}{|c|c|c|c|c|c|c|}
\hline
\textbf{SEMESTRE} & \textbf{SEG} & \textbf{TER} & \textbf{QUA} & \textbf{QUI} & \textbf{SEX} & \textbf{SÁB} \\ \hline
1º & 18 & 18 & 18 & 17 & 15 & 15 \\ \hline
2º & 16 & 17 & 17 & 18 & 17 & 17 \\ \hline
\multicolumn{6}{|c|}{\small \textbf{Total de dias efetivos de atividades acadêmicas:}}              & 209            \\ \hline
\end{tabular}
}
\end{table}
\newline
\null
\newline
\end{center}
\vfill
\null
\columnbreak
~
\vfill
\begin{center}
\large \textbf{DATAS IMPORTANTES PARA OS CURSOS DO CALEM}
\newline
\null
\newline
\begin{table}
\centering
\resizebox{0.9\columnwidth}{!}{
\begin{tabular}{|c|c|c|c|c|}
\hline
\textbf{SEMESTRE} & \textbf{\begin{tabular}[c]{@{}c@{}}1ª NOTA\\ BIMESTRAL\end{tabular}} & \textbf{\begin{tabular}[c]{@{}c@{}}2ª NOTA \\ BIMESTRAL\end{tabular}} & \textbf{\begin{tabular}[c]{@{}c@{}}RESULTADOS\\ FINAIS\end{tabular}} & \textbf{\begin{tabular}[c]{@{}c@{}}DIÁRIOS DE\\ CLASSE - DERAC\end{tabular}} \\ \hline
1º & 1 & 00 & 00 & 00 \\ \hline
2º & 00 & 00 & 00 & 00 \\ \hline
\end{tabular}
}
\end{table}
\newline
\null
\newline
\end{center}
\vfill
\null
\newpage\onespacing \section{\color{hmcorange}Feriados, Recessos e Férias}\subsection{Janeiro}\textbf{01}\quad \quad \quad \quad \textbf{Confraternização Universal} \newline\textbf{02 a 31}\quad \quad Férias Docentes \newline\subsection{Fevereiro}\textbf{24 a 25}\quad \quad \textbf{Carnaval} \newline\textbf{26}\quad \quad \quad \quad Recesso (Cinzas) \newline\subsection{Abril}\textbf{10}\quad \quad \quad \quad \textbf{Paixão de Cristo} \newline\textbf{11}\quad \quad \quad \quad Recesso (Sábado de Aleluia) \newline\textbf{12}\quad \quad \quad \quad \textbf{Páscoa} \newline\textbf{20}\quad \quad \quad \quad Recesso \newline\textbf{21}\quad \quad \quad \quad \textbf{Tiradentes} \newline\subsection{Maio}\textbf{01}\quad \quad \quad \quad \textbf{Dia Mundial do Trabalho} \newline\textbf{02}\quad \quad \quad \quad Recesso \newline\subsection{Junho}\textbf{11}\quad \quad \quad \quad \textbf{\textit{Corpus Christi}} \newline\textbf{12 a 13}\quad \quad Recesso \newline\subsection{Julho}\textbf{14}\quad \quad \quad \quad \textbf{Término do primeiro semestre de 2020} \newline\textbf{17 a 31}\quad \quad Férias Docentes \newline\subsection{Setembro}\textbf{07}\quad \quad \quad \quad \textbf{Independência do Brasil} \newline\textbf{08}\quad \quad \quad \quad Padroeiro de Curitiba \newline\subsection{Outubro}\textbf{12}\quad \quad \quad \quad \textbf{Padroeira do Brasil} \newline\textbf{28}\quad \quad \quad \quad \textbf{Dia do Servidor Público} \newline\subsection{Novembro}\textbf{02}\quad \quad \quad \quad \textbf{Finados} \newline\textbf{15}\quad \quad \quad \quad \textbf{Proclamação da República} \newline\subsection{Dezembro}\textbf{17}\quad \quad \quad \quad \textbf{Término do segundo semestre de 2020} \newline\textbf{22 a 23}\quad \quad Recesso Acadêmico \newline\textbf{24}\quad \quad \quad \quad Recesso \newline\textbf{25}\quad \quad \quad \quad \textbf{Natal} \newline\textbf{26}\quad \quad \quad \quad Recesso \newline\textbf{28 a 30}\quad \quad Recesso Acadêmico \newline\textbf{31}\quad \quad \quad \quad Recesso \newline\newpage
\section{\color{hmcorange}Atividades E Eventos}\subsection{Abril}\textbf{01}\quad \quad \quad \quad Exemplo \newline\subsection{Maio}\textbf{04}\quad \quad \quad \quad ex2 \newline\subsection{Agosto}\textbf{08 a 11}\quad \quad agosto \newline\end{poster}
\end{document}